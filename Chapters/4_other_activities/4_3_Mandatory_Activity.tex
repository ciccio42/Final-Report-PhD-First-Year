\section{Mandatory requirements}

\subsection{International Conference as presenting author and Paper as corresponding author}
In the second year of our project, we did not produce any scientific publications for either conferences or journals. Nonetheless, because our initial findings presented intriguing insights, we are in the process of preparing a paper for the ``European Robotic Forum 2024" (ERF 2024), scheduled to take place in Rimini, Italy from March 13 to 15, 2024. The submission deadline for scientific materials is set for November 1, 2024.
In addition, we have intentions to expand and validate our current experiments, with the aim of creating another publication in reputable journals like ``IEEE Robotics and Automation Letters" (RA-L).

\subsection{Period Abroad}
I will spent the period abroad next year. We are currently assessing various opportunities with tutors who can enhance my skills, focusing on areas that complement what I've learned so far. These tutors will also provide valuable feedback on my research in AI-Enabled Robotics. Specifically, we are considering the tutorship with the following professors:
\begin{enumerate}
    \item \textit{Alberto Sanfeliu}, director of the Artificial Vision and Intelligent System Group (VIS) at the ``Universitat Polit`ecnica de Catalunya - Barcelona (Spain)" (UPC).  His research interests include intelligent robotics, human–robot interactions, computer vision, and pattern recognition.
    \item \textit{Francesc Serratosa}, full professor of Computer Science at the ``Universitat Rovira i Virgili - Tarragona (Spain)''. He appears in the list of the 2\% most influential researchers in the world, presented in ``Updated science-wide author databases of standardized citation indicators"\cite{}. He is active in research in the areas of Computer Vision, Machine Learning and Artificial Intelligence.
    \item \textit{Xiaoyi Jiang}, full professor of Computer Science at the University of Munster - Munster (Germany). His current research interests include 3D image analysis, and structural pattern recognition.
\end{enumerate}