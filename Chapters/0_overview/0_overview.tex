\clearpage
%\section*{\center Abstract}
%\vspace{5 ex}
\chapter*{Overview}\label{chapter:overview}
\addcontentsline{toc}{chapter}{Overview}

The report represents the final document of the first year of the doctorate carried out by the undersigned. The document conforms to the latest version of the document \hyperlink{https://corsi.unisa.it/ingegneria-dell-informazione/didattica/guida-dello-studente}{\textit{\say{Requirements for the PhD degree}}} approved on 14/12/2020 by the professors' college and available on the university platform in the area reserved for doctoral students. As requested, the document contains a report on the activities carried out by the student during the first year. \par
The document is divided, as suggested in the aforementioned document, into 3 mandatory chapters, Background \ref{chapter:background}, Research Plan \ref{chapter:research_plan}, and Other Activities \ref{chapter:other_activities}.
\newline The first chapter contains the Introduction (Section \ref{sec:intro}) and the State-of-the-Art (Section \ref{sec:sota}) sections. The former introduces to the topic of \textit{Learning from Demonstration}, that is the main topic of this report, pointing out the the importance of the field and the existing knowledge. The latter presents the literature review. This section analyzes from a methodological and application perspective the main approaches by which the problem is solved, highlighting the pros and cons of each method.
\newline The second chapter containes the Research Gaps (Section \ref{sec:research_gaps}) and the Research Plan (Section \ref{sec:research_activity}) sections. The first presents the gaps that can be extrapolated from the State-of-the-Art, with respect to the context of interest. The second reports the proposed research activity, going on to highlight the connection between the proposal and the gaps presented earlier, the importance of the proposal with respect to the context of interest, and the methodological and experimental procedures that will be followed to track and evaluate progress.
\newline The third chapter shows all the activities carried out during the year, including courses attended at the university, both mandatory and not.
% \begin{adjustwidth}{38pt}{38pt}
% \textit{Here I write the abtract.}
% \end{adjustwidth}