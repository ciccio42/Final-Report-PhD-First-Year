\paragraph{Generative Adversarial Imitation Learning (GAIL) [MAX 5]} \mbox{} \\
Generative Adversarial Imitation Learning was proposed for the first time in \cite{ho2016gail}, with the idea to improve the IRL setting, which is expensive to run, because of the double-nested optimization procedure. The authors in \cite{ho2016gail}, starting from a general Max-Ent formulation (Formula \ref{formula:regularized_max_ent}), obtained a characterization of the learned policy (Formula \ref{formula:policy_characterization}), where $\psi(c)$ is a cost-regularizer, $\psi^{*}(c)$ is its conjugate, and $\rho_{\pi}$ is the \textit{occupancy measure}, i.e. the distribution of state-action pairs that the agent encounters when navigating the environment with policy $\pi$. The interpretation of Formula \ref{formula:policy_characterization} is that the $\psi-regularized$ IRL finds a policy whose occupancy measure is similar to the expert's one, measured by $\psi^{*}$. The next-step was to choose an appropriate regularization function. In particular, by choosing the regularizer in Formula \ref{formula:ga_regularization}, the conjugate in Formula \ref{formula:ga_regularizer_conjugate} can be obtained, which is the classic Adversarial-Learning Loss, where the current policy $\pi^{L}$ plays the role of GAN generator, and $D$ is the GAN discriminator, which has to distinguish between state-action pairs generated either by the expert-policy or by the current policy. 
\begin{equation}
    \label{formula:regularized_max_ent}
    IRL_{\psi}(\pi^{E}) = \underset{c \in R^{S \times A}}{arg \ max} - \psi(c) +  (\underset{\pi^{L} \in \Pi}{\min} -\mathcal{H}(\pi^{L}) + \mathbb{E}_{\pi^{L}} \left [ c(s,a) \right ]) - \mathbb{E}_{\pi^{E}} \left [ c(s,a) \right ]
\end{equation}

\begin{equation}
    \label{formula:policy_characterization}
    RL \circ IRL_{\psi}(\pi^{E}) = \underset{\pi^{L} \in \Pi}{arg \ min}-\mathcal{H}(\pi^{L}) + \psi^{*}(\rho_{\pi^{L}} - \rho_{\pi^{E}}) 
\end{equation}

\begin{equation}
    \label{formula:ga_regularization}
    \psi_{GA}(c) = \left\{\begin{matrix}
        \mathbb{E}_{\pi^{E}}\left [ g(c(s,a)) \right ] &  if \ c < 0\\ 
        + \infty & otherwise
        \end{matrix}\right., \  g(x) = \left\{\begin{matrix}
                        -x - log(1- e^{x}) &  if \ c < 0\\ 
                        + \infty & otherwise
                        \end{matrix}\right.
\end{equation}

\begin{equation}
    \label{formula:ga_regularizer_conjugate}
    \psi^{*}_{GA}(\rho_{\pi^{L}} - \rho_{\pi^{E}}) = \underset{D\in(0,1)^{S \times A}}{max} \mathbb{E}_{\pi^{L}}\left [ \log(D(s,a))\right ] +\mathbb{E}_{\pi^{E}}\left [ \log(1 - D(s,a))\right ]
\end{equation}
