\paragraph{Generative Adversarial Imitation Learning (GAIL) [MAX 5]} \mbox{} \\
Generative Adversarial Imitation Learning was proposed for the first time in \cite{ho2016gail}, with the idea to improve the IRL setting, which is expensive to run, because of the double-nested optimization procedure. The authors in \cite{ho2016gail}, starting from a general Max-Ent formulation (Formula \ref{formula:regularized_max_ent}), obtained a characterization of the learned policy (Formula \ref{formula:policy_characterization}), where $\psi(c)$ is a cost-regularizer, $\psi^{*}(c)$ is its conjugate, and $\rho_{\pi}$ is the \textit{occupancy measure}, i.e. the distribution of state-action pairs that the agent encounters when navigating the environment with policy $\pi$. The interpretation of Formula \ref{formula:policy_characterization} is that the $\psi-regularized$ IRL finds a policy whose occupancy measure is similar to the expert's one, measured by $\psi^{*}$. The next-step was to choose an appropriate regularization function. In particular, by choosing the regularizer in Formula \ref{formula:ga_regularization}, the conjugate in Formula \ref{formula:ga_regularizer_conjugate} can be obtained, which is the classic Adversarial-Learning Loss, where the current policy $\pi^{L}$ plays the role of GAN generator, and $D$ is the GAN discriminator, which has to distinguish between state-action pairs generated either by the expert-policy or by the current policy.\begin{equation}
    \label{formula:regularized_max_ent}
    IRL_{\psi}(\pi^{E}) = \underset{c \in R^{S \times A}}{arg \ max} - \psi(c) +  (\underset{\pi^{L} \in \Pi}{\min} -\mathcal{H}(\pi^{L}) + \mathbb{E}_{\pi^{L}} \left [ c(s,a) \right ]) - \mathbb{E}_{\pi^{E}} \left [ c(s,a) \right ]
\end{equation}

\begin{equation}
    \label{formula:policy_characterization}
    RL \circ IRL_{\psi}(\pi^{E}) = \underset{\pi^{L} \in \Pi}{arg \ min}-\mathcal{H}(\pi^{L}) + \psi^{*}(\rho_{\pi^{L}} - \rho_{\pi^{E}}) 
\end{equation}

\begin{equation}
    \label{formula:ga_regularization}
    \psi_{GA}(c) = \left\{\begin{matrix}
        \mathbb{E}_{\pi^{E}}\left [ g(c(s,a)) \right ] &  if \ c < 0\\ 
        + \infty & otherwise
        \end{matrix}\right., \  g(x) = \left\{\begin{matrix}
                        -x - log(1- e^{x}) &  if \ c < 0\\ 
                        + \infty & otherwise
                        \end{matrix}\right.
\end{equation}

\begin{equation}
    \label{formula:ga_regularizer_conjugate}
    \psi^{*}_{GA}(\rho_{\pi^{L}} - \rho_{\pi^{E}}) = \underset{D\in(0,1)^{S \times A}}{max} \mathbb{E}_{\pi^{L}}\left [ \log(D(s,a))\right ] +\mathbb{E}_{\pi^{E}}\left [ \log(1 - D(s,a))\right ]
\end{equation}
Based on these considerations the Algorithm \ref{alg:gail} has been proposed. The algorithm comprises two fundamental steps, the first related to the Discriminator's parameter update and the second related to the policy's parameter update. Since GAIL has proven to be more effective than classic IRL algorithm \cite{ziebart2008maximum_entropy}, subsequent works have focused either on improving the sample efficiency, by replacing the model-free on-policy TRPO algorithm, with an off-policy RL algorithm, such as in \cite{kostrikov2018discriminator}, or by modifying the reward function in input to the RL algorithm \cite{fu2018airl,ghasemipour2020divergence_minimization_perspective}.\begin{algorithm}[tb]
\caption{Generative Adversarial Imitation Learning Algorithm}\label{alg:gail}
\begin{algorithmic}
\Require Expert Trajectories $\tau^{E} \sim \pi^{E}$, initial policy $\pi^{L}_{\theta}$, discriminator $D_{\omega}$
\For {$i=1, \dots, N$} 
    \State Sample trajectories, $\tau^{L}_{i} \sim \pi^{L}_{\theta}$
    \State Update Discriminator, $\mathbb{\hat{E}}_{\tau^{L}_{i}}\left [\nabla_{\omega} \log(D_{\omega}(s,a))\right ] +\mathbb{\hat{E}}_{\tau^{E}}\left [\nabla_{\omega} \log(1 - D_{\omega}(s,a))\right ]$
    \State Update Policy $\pi_{\theta}$, with TRPO \cite{schulman2015trpo}, and cost-function $C(s,a)=\log(D_{\omega}(s,a))$ 
\EndFor
\end{algorithmic}
\end{algorithm}All the cited methods have reported promising results on control tasks in a simulation environment \cite{brockman2016openai}, but they worked in a low-dimensional state-space, indeed when the GAIL method was adapted to work with a high-dimensional state-space, like in \cite{liu2018imitation_from_observation,reddy2019sqil,zolna2021task_relevant_ail,rafailov2021visual_ail}, it shown very poor results. In particular, with respect to the Adversarial Imitation Learning setting, works of interest are \cite{zolna2021task_relevant_ail,rafailov2021visual_ail}. In \cite{zolna2021task_relevant_ail}, the authors focused on solving the \textbf{casual-confusion} problem. This problem occurs when the discriminator, during the learning process, focuses on task-irrelevant features between expert and policy generated transactions, this causes the rewards to become uninformative. To reduce the casual-confusion problem, in \cite{zolna2021task_relevant_ail} two elements have been proposed: \begin{enumerate*}[label=(\textbf{\arabic*})]
    \item A regularization term, with the aim to make the discriminator \textbf{unable} to distinguish between constraining sets $I_{E}$ and $I_{A}$. The constraining sets are composed of expert and agent observations, such that a sample can belong either to $I_{E}$ or $I_{A}$, based on spurious features.  
    \item An early-stopping policy, Actor Early-Stopping (AES), that restarts the episode if the discriminator score at the current step exceeds the median score of the episode so far for $T_{patience}$ consecutive steps.
\end{enumerate*}. The regularization term was defined as
$\psi = \frac{1}{2} \ \mathbb{E}_{s \in \mathcal{I}_{E}} \left[ \mathbf{1}_{D_{\omega} \geq  \frac{1}{2}}\right] + \frac{1}{2} \ \mathbb{E}_{s \in \mathcal{I}_{A}} \left[ \mathbf{1}_{D_{\omega} <  \frac{1}{2}}\right]$
