\newpage
\subsection{Summary}
\label{sec:summary}
This section summarizes all the approaches seen so far, trying to understand their pros and cons. Answering which approaches leads to better performance than the others is not trivial for several reasons ranging from the fact that, as mentioned above, methods are tested in different environments (e.g., some in simulation, others in the real world), on different tasks and using different robotic platforms. It can be said that, by analyzing the state of the art, the most studied approach is Behavioral Cloning. This approach, from a pure implementation point of view, is the simplest, as, in principle, it allows the execution of a purely offline training procedure. Unfortunately, some problems have to be handled, ranging from compounding-error to dataset creation. About the former, some solutions have been proposed, based on interactive learning algorithms, which reduce the covariate-shift phenomena, but introduce the need for active human supervision during the learning. During the past few years, there has been increasing interest in Meta-Learning algorithms, with the ultimate goal of achieving a system that can generalize across different tasks. However, as it turns out, the generalization problem is far from solved. 
Regarding approaches such as IRL, GAIL, and LfO. Compared to BC, methods such as IRL, GAIL, and LfO are more efficient regarding required demonstrations. However, they introduce RL algorithms into the optimization loop, which requires interaction with the environment, which can be risky and time-consuming. Moreover, despite their promising results, relatively few methods of these three approaches have actually been used in real-world vision-based manipulation tasks, effectively leaving unanswered the question of whether these methods can be used.