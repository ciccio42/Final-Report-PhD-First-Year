\subsection{Summary}
\label{sec:summary}
This section summarizes all the approaches seen so far, trying to understand their pros and cons. Answering which approaches leads to better performance than the others is not trivial for several reasons ranging from the fact that, as mentioned above, methods are tested in different environments (e.g., some in simulation, others in the real world) on different tasks and using robotic platforms. It can be said that analyzing the state of the art and that the most studied approach is behavioral cloning, particularly in recent years, there has been a focus on meta-learning methods. This approach, from a pure implementation point of view, is the simplest, as, in principle, it allows the execution of a purely offline training procedure. Unfortunately, some problems have to be handled, ranging from compound-error to dataset creation. Regarding approaches such as IRL, GAIL, and LfO. The most interesting ones are GAIL and LfO. Regarding GAIL, the proposed methods are. Compared to BC, methods such as IRL, GAIL, and LfO are more efficient regarding required demonstrations. However, they introduce RL algorithms into the optimization loop, which requires interaction with the environment, which can be risky and time-consuming. Moreover, despite their promising results, relatively few methods of these three approaches have actually been used in real-world vision-based manipulation tasks, effectively leaving unanswered the question of whether these methods can be used.