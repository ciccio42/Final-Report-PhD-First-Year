\section{Proposed Research Activity}
\label{sec:research_activity}
In this section the proposed research activity will be presented, and linked with the gaps discussed in Section \ref{sec:research_gaps}. 
\newline The final goal of this research activity is to design and develop an innovative Artificial-Intelligence based system that can allow a robotic platform to execute tasks starting from a set of demonstrations. The focus will be posted on different aspects, organized in such a way that they will bring to the development of a complete AI system. 
\paragraph{Dataset} \mbox{} \\
%When approaching such systems, the first aspect to consider is related to dataset creation and how demonstrations are defined. As seen, there are two main procedures, the first being direct control of the manipulator by teleoperation and the second being the use of an indirect technique based solely on vision systems. Gaps concerning the dataset are related to:
%\begin{enumerate*}[label=\textbf{(\arabic*)}]
%    \item the lack of multi-modal sources of information;
%    \item the possibility to reause a publicly available dataset.
%\end{enumerate*} 
%With respect to these points, the proposal is related to the development of a data collection procedure 
\paragraph{Architecture} \mbox{} \\
%The core of these learning-based systems is the DL architecture. The information processing performed by th 
%The idea is to develop innovative DL architectures capable of taking a step forward in the current literature. Thus capable of:
%\begin{enumerate}
%    \item Efficiently and effectively process multimodal information sources (e.g., visual and proximity/tactile) to create a complete representation of the environment, and compensate for errors generated by the failure of a particular source (e.g., object occlusion due to the arm movement);
%    \item Explicitly model spatio-temporal information, which can be exploited for a better inference of the action to be performed; 
%    \item Efficiently and effectively extrapolate possible task-relevant features from a high-level task demonstration (e.g., human performing a task, manipulated object trajectories).  
%\end{enumerate}
\paragraph{System deployment} \mbox{} \\
%\dots
%Progress in each field will be assessed first of all in simulation environments, then on real robotic platform. 