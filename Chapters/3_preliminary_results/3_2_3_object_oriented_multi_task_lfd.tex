\subsection{Object Oriented Multi-Task Learning from Demonstration}
\label{sec:obj_oriented_multi_task_lfd}
As discussed in Section \ref{sec:baseline_definition}, the system is capable of generating valid trajectories but struggles to select the correct object. To test the hypothesis that enhancing the control module with low-level information, such as the target object's position, would simplify the control task and lead to improved performance, we trained a model similar to the one proposed in \cite{mandi2022towards_more_generalizable_one_shot}, but giving in input to the control module also the information about the target-object position, rapresented by the bounding box coordinates.
Consequently the novel system can be defined as a \textit{Conditioned-Policy} $\pi^{L}(a_{t} | o_{t}, d_{task}, bb_{t})$ that generates the current action $a_{t}$ based on 3 inputs: \begin{enumerate*}[label=(\arabic*)]
    \item The current observation $o_{t}$;
    \item The task description $d_{task}$;
    \item The target-object bounding box $bb_{t}$.
\end{enumerate*}
\newline Table \ref{table:performance_with_bounding_box}, reportes the results obtained on the two tasks pick-place and nut-assembly, comparing the results obtained in Section \ref{sec:baseline_definition}, with two different conditions:
\begin{enumerate*}[label=(\arabic*)]
    \item The control module receives in input the ground truth bounding box;
    \item The control module receives in input the bounding box predicted with the module designed in Section \ref{sec:cond_target_obj_detector}.
\end{enumerate*}
\newline Table \ref{table:performance_with_bounding_box} reports the results obtained by adding the low-level bounding-box information to the control module.
\begin{table}[htb]
    \centering
    \fontsize{11pt}{11pt}
    \selectfont
    \caption{Performance comparison on pick-place and nut-assembly tasks, adding the information about the target object position}
    \label{table:performance_with_bounding_box}
    \resizebox{\linewidth}{!}{%
        \begin{tabular}{>{\centering\hspace{0pt}}m{0.192\linewidth}>{\centering\hspace{0pt}}m{0.35\linewidth}>{\centering\hspace{0pt}}m{0.142\linewidth}>{\centering\hspace{0pt}}m{0.115\linewidth}>{\centering\arraybackslash\hspace{0pt}}m{0.131\linewidth}}
            \hline
            \textbf{Task}                             & \textbf{Setup}             & \begin{tabular}[c]{@{}c@{}}\textbf{Reaching} \\\textbf{Rate }\\{[}\%]\end{tabular} & \begin{tabular}[c]{@{}c@{}}\textbf{Picking}\\\textbf{Rate }\\{[}\%]\end{tabular} & \begin{tabular}[c]{@{}c@{}}\textbf{Success}\\\textbf{Rate}\\{[}\%]\end{tabular} \\
            \hline
            \multirow{3}{*}{\Centering{}Pick-Place}   & Baseline                   & 66.8                                                                                                     & 64.3                                                                                                   & 58.7                                                                                                  \\
                                                      & Baseline with GT bb        & 100                                                                                                      & 96.5                                                                                                   & 76.8                                                                                                  \\
                                                      & Baseline with predicted bb & \textbf{97.5}                                                                                            & \textbf{92.5}                                                                                          & \textbf{78.1}                                                                                         \\
            \hline
            \multirow{3}{*}{\Centering{}Nut-Assembly} & Baseline                   & 40.0                                                                                                     & 38.8                                                                                                   & 36.6                                                                                                  \\
                                                      & Baseline with GT bb        & 100.0                                                                                                    & 98.8                                                                                                   & 60.0                                                                                                  \\
                                                      & Baseline with predicted bb & 3\textbf{8.8}                                                                                            & \textbf{38.8}                                                                                          & \textbf{26.6}                                                                                         \\
            \hline
        \end{tabular}
    }
\end{table}
As it can be noted, we obtain a relevant improvement in both the reaching-rate, since the robot is always able to reach the target object identified by the predicted bounding-box. Proving that, learn to reach a target object starting from a low-level information (e.g., bounding-box) is easier to learn to reach the same target starting from an high-level information (e.g., command and current observation). We obtain also a relevant improvement in success-rate for pick-place task. Whith the system that shows novel error distributions reported in Table \ref{table:baseline_with_bb_error_cases}.
\begin{table}[bth!]
    \caption{Relevant error cases in pick-place task (\ref{table:baseline_with_bb_pp}) and nut-assembly task (\ref{table:baseline_with_bb_na})}
    \label{table:baseline_with_bb_error_cases}
    \begin{subtable}[h]{0.45\textwidth}
        \centering
        \fontsize{9pt}{9pt}
        \selectfont
        \caption{Relevant error cases in pick-place task}
        \label{table:baseline_with_bb_pp}
        \resizebox{\linewidth}{!}{%
            \begin{tblr}{
                cells = {c},
                hlines,
                hline{1,5} = {-}{0.08em},
                    }
                \textbf{Error case} & \textbf{\#Occurrences} \\
                Wrong bin           & 17                     \\
                Fail to pick        & 9                      \\
                Other               & 8
            \end{tblr}
        }
    \end{subtable}
    \hfill
    \begin{subtable}[h]{0.45\textwidth}
        \centering
        \fontsize{9pt}{9pt}
        \selectfont
        \caption{Relevant error cases in nut-assembly task}
        \label{table:baseline_with_bb_na}
        \resizebox{\linewidth}{!}{%
            \begin{tblr}{
                cells = {c},
                hlines,
                hline{1,5} = {-}{0.08em},
                    }
                \textbf{Error case} & \textbf{\#Occurrences} \\
                -                   & -                      \\
                -                   & -                      \\
                -                   & -
            \end{tblr}
        }
    \end{subtable}

\end{table}
\newline \textbf{TO COMPLETE WITH LAST CONSIDERATIONS}